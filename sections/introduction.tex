\section{Introduction}

This paper is for the introduction of the C programming language, and its usage in the linux kernel.
Reading this you will also learn how to create linux kernel modules, and communicate between them and user applications.

Please not that, altaugh this is ment to be an introductory paper and course, this paper will assume that you have
previous programming experience, and know how to naviagate the linux filesystem.
Therefore it will not contain full explanation for basic linux commands and howtos, only examples.

Basic environment setup will be provided.


\subsection{Prerequirerities}

First we will need a linux machine, which can be obtained trough many ways.
You can eather use a Virtual Box, your own machine, or WSL.
\textit{It is advised to use a VM leter on the course, as crashing the kernel when programming modules is common.}

Lets see how to enable WSL on Windows: % todo cite..

\begin{lstlisting}[style=CStyle,caption={Installing WSL}]
PS > wsl --install                      # Enable the windows linux subsystem
PS > wsl --list --online                # List all available VMs
PS > wsl --install -d Ubuntu            # Install an Ubuntu virtual machine

\end{lstlisting}

From here on out, the following commands will assume, that you are using Debian/Ubuntu.
If that is not the case, \textit{I assume you have the proper knowledge of how to use your own system/distribution}.

\vspace{1em}
\textbf{We will need to following programs trough the course:}
\begin{itemize}[noitemsep]
    \item gcc or clang % todo cite..
    \item make and kernel-make % todo cite..
    \item cmake % todo cite..
\end{itemize}

To install them, run:
\begin{lstlisting}[style=CStyle,caption={Installing WSL}]
$ sudo apt install gcc clang make cmake build-essential kmod

\end{lstlisting}
