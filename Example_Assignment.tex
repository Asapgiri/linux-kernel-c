\documentclass[12pt,a4paper]{article}
\usepackage{styles/oeNIKstyle}
\usepackage{styles/ngk813_style}
\usepackage{array}
\usepackage{hyperref}

\usepackage{t1enc}
\usepackage{url}
\usepackage{pdfpages}

\setlength{\parindent}{0 mm} % behúzás mértékét állíthatjuk be

% megadandó adatok
\author{Korcsák Gergely}

% Beállítások
% első változó: a számozás helye 1 = jobb oldal, 2 = középen
% második változó: mutassa-e az összes oldalt. 0 = nem, 1 = igen.
\setPageNumbering{1}{0}

\begin{document}

    % lstlisting numbering
    \renewcommand{\thelstlisting}{\arabic{section}.\arabic{lstlisting}}
    \renewcommand{\lstlistoflistings}{\begingroup
        \tocfile{Code snippets}{lol}
    \endgroup}

    \setcounter{page}{1} % start counting pages from here
    \setalgorithmcounter % set up algorithm numbering

    % start document here...

    {
        \centering
        \Large
        \textbf{Example assignments}
        \hline
        \vspace{1em}
    }

    \begin{enumerate}
        \item Create a chat application between 2 separate computers. Communicate trough the internet with using a linux
            kernel module. The module should act like a server. It can accept messages, and also send them to a specific
            IP address. (The port can be defined as a constant)

            The user-space application should communicate with the server trough a char device.
            The user application should show messages in "real time", and also send messages, using the server.

            Implementation can also use the proc functions, for debugging, or other purposes.

        \item Create an application that will driver a robot arm using ROS. The kernel module should send the robot arm
            to specific locations. And keep it there. (Both simple functins and path calculation are welcome)

            The user-space application should set new coordinates for the robot using char devices.
            Bonus point, if it can also send trough the rebot in a specific path that is defined by parameters in a file.

            ++Extra point if the application also thinks about the surfaces, that the robot can collide with to prevent
            the collisions, and also draw or do a niche thing.
    \end{enumerate}



\end{document}
