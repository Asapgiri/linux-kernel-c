%----------------------------------------------------------------------------------------
%	PACKAGES AND OTHER DOCUMENT CONFIGURATIONS
%----------------------------------------------------------------------------------------

\documentclass[
	11pt, % Set the default font size, options include: 8pt, 9pt, 10pt, 11pt, 12pt, 14pt, 17pt, 20pt
	%t, % Uncomment to vertically align all slide content to the top of the slide, rather than the default centered
	%aspectratio=169, % Uncomment to set the aspect ratio to a 16:9 ratio which matches the aspect ratio of 1080p and 4K screens and projectors
]{beamer}
\geometry{mag=2000,truedimen}

\graphicspath{{Images/}{./}}

\usepackage{booktabs}

\usetheme{Madrid}
\usecolortheme{orchid}
\usefonttheme{default} % Typeset using the default sans serif font

\usepackage{palatino} % Use the Palatino font for serif text
\usepackage[default]{opensans} % Use the Open Sans font for sans serif text

\useinnertheme{rectangles}

%\usepackage{styles/ngk813_style}

% ------------------------------------------------------------------------------
% Listings
% ------------------------------------------------------------------------------
\usepackage{textcomp}
\usepackage{listings}
\usepackage{inconsolata}
\usepackage{fontawesome} % egyedi listajelekhez például
\usepackage{graphicx}

\definecolor{mygreen}{HTML}{37980D}
\definecolor{myblue}{HTML}{0D089F}
\definecolor{myred}{HTML}{98290D}
\definecolor{mGreen}{rgb}{0,0.6,0}
\definecolor{mGray}{rgb}{0.5,0.5,0.5}
\definecolor{mPurple}{rgb}{0.58,0,0.82}
\definecolor{backgroundColour}{rgb}{0.98,0.98,0.98}

\newcommand{\quotetext}[1]{\begin{block}\footnotesize\textit{#1}\end{block}}

\newcommand{\sft}[1]{\section{#1}\frametitle{#1}\centering\LARGE#1}
\newcommand{\ssft}[1]{\subsection{#1}\frametitle{#1}}

\newcommand{\ascode}[1]{\textcolor{blue}{\fontfamily{DejaVu Sans Mono}\ttfamily\selectfont\replunderscores{#1}}}
\newcommand{\bfcode}[1]{\textbf{\ascode{#1}}}
\newcommand{\replunderscores}[1]{\expandafter\@repl@underscores#1_\relax}
\def\@repl@underscores#1_#2\relax{%
    \ifx \relax #2\relax
        % #2 is empty => finish
        #1%
    \else
        % #2 is not empty => underscore was contained, needs to be replaced
        #1%
        \textunderscore
        % continue replacing
        % #2 ends with an extra underscore so I don't need to add another one
        \@repl@underscores#2\relax
    \fi
}

\def\lstbasicfont{\fontfamily{DejaVu Sans Mono}\selectfont\footnotesize}

\lstset{%
basicstyle=\ttfamily
}
\lstdefinestyle{CStyle}{
    backgroundcolor=\color{backgroundColour},
    commentstyle=\color{mGreen},
    keywordstyle=\color{blue},
    numberstyle=\tiny\color{mGray},
    stringstyle=\color{mPurple},
    basicstyle=\footnotesize\ttfamily,
    breakatwhitespace=false,
    breaklines=true,
    captionpos=b,
    keepspaces=true,
    numbers=left,
    numbersep=5pt,
    showspaces=false,
    showstringspaces=false,
    showtabs=false,
    tabsize=4,
    language=C,
    xleftmargin=0.5cm,
    xrightmargin=0.5cm,
    upquote=true,
    inputencoding = utf8,  % Input encoding
    extendedchars = true,  % Extended ASCII
    literate      =        % Support additional characters
      {á}{{\'a}}1  {é}{{\'e}}1  {í}{{\'i}}1 {ó}{{\'o}}1  {ú}{{\'u}}1
      {Á}{{\'A}}1  {É}{{\'E}}1  {Í}{{\'I}}1 {Ó}{{\'O}}1  {Ú}{{\'U}}1
      {à}{{\`a}}1  {è}{{\`e}}1  {ì}{{\`i}}1 {ò}{{\`o}}1  {ù}{{\`u}}1
      {À}{{\`A}}1  {È}{{\'E}}1  {Ì}{{\`I}}1 {Ò}{{\`O}}1  {Ù}{{\`U}}1
      {ä}{{\"a}}1  {ë}{{\"e}}1  {ï}{{\"i}}1 {ö}{{\"o}}1  {ü}{{\"u}}1
      {ő}{{\~o}}1  {Ő}{{\~O}}1  {ű}{{\~u}}1 {Ű}{{\~U}}1
      {Ä}{{\"A}}1  {Ë}{{\"E}}1  {Ï}{{\"I}}1 {Ö}{{\"O}}1  {Ü}{{\"U}}1
      {â}{{\^a}}1  {ê}{{\^e}}1  {î}{{\^i}}1 {ô}{{\^o}}1  {û}{{\^u}}1
      {Â}{{\^A}}1  {Ê}{{\^E}}1  {Î}{{\^I}}1 {Ô}{{\^O}}1  {Û}{{\^U}}1
      {œ}{{\oe}}1  {Œ}{{\OE}}1  {æ}{{\ae}}1 {Æ}{{\AE}}1  {ß}{{\ss}}1
      {ç}{{\c c}}1 {Ç}{{\c C}}1 {ø}{{\o}}1  {Ø}{{\O}}1   {å}{{\r a}}1
      {Å}{{\r A}}1 {ã}{{\~a}}1  {õ}{{\~o}}1 {Ã}{{\~A}}1  {Õ}{{\~O}}1
      {ñ}{{\~n}}1  {Ñ}{{\~N}}1  {¿}{{?`}}1  {¡}{{!`}}1
      {°}{{\textdegree}}1 {º}{{\textordmasculine}}1 {ª}{{\textordfeminine}}1
      % ¿ and ¡ are not correctly displayed if inconsolata font is used
      % together with the lstlisting environment. Consider typing code in
      % external files and using \lstinputlisting to display them instead.
}

\lstdefinestyle{CMake}{
    backgroundcolor=\color{backgroundColour},
    commentstyle=\color{gray},
    numberstyle=\tiny\color{mGray},
    stringstyle=\color{mPurple},
    basicstyle=\footnotesize\ttfamily,
    breakatwhitespace=false,
    breaklines=true,
    captionpos=b,
    keepspaces=true,
    numbers=left,
    numbersep=5pt,
    showspaces=false,
    showstringspaces=false,
    showtabs=false,
    tabsize=4,
    language=bash,
    classoffset=0,
    morekeywords={cmake_minimum_required,project,include_directories,add_executable,CMAKE_RUNTIME_OUTPUT_DIRECTORY,CMAKE_BINARY_DIR,CMAKE_CURRENT_SOURCE_DIR},
    keywordstyle=\color{mGreen},
    classoffset=1,
    morekeywords={VERSION,},
    keywordstyle=\color{orange},
    classoffset=2,
    morekeywords={\$,\{,\}},
    keywordstyle=\color{myblue},
    xleftmargin=0.5cm,
    xrightmargin=0.5cm,
    upquote=true,
    inputencoding = utf8,  % Input encoding
    extendedchars = true,  % Extended ASCII
    literate      =        % Support additional characters
      {á}{{\'a}}1  {é}{{\'e}}1  {í}{{\'i}}1 {ó}{{\'o}}1  {ú}{{\'u}}1
      {Á}{{\'A}}1  {É}{{\'E}}1  {Í}{{\'I}}1 {Ó}{{\'O}}1  {Ú}{{\'U}}1
      {à}{{\`a}}1  {è}{{\`e}}1  {ì}{{\`i}}1 {ò}{{\`o}}1  {ù}{{\`u}}1
      {À}{{\`A}}1  {È}{{\'E}}1  {Ì}{{\`I}}1 {Ò}{{\`O}}1  {Ù}{{\`U}}1
      {ä}{{\"a}}1  {ë}{{\"e}}1  {ï}{{\"i}}1 {ö}{{\"o}}1  {ü}{{\"u}}1
      {ő}{{\~o}}1  {Ő}{{\~O}}1  {ű}{{\~u}}1 {Ű}{{\~U}}1
      {Ä}{{\"A}}1  {Ë}{{\"E}}1  {Ï}{{\"I}}1 {Ö}{{\"O}}1  {Ü}{{\"U}}1
      {â}{{\^a}}1  {ê}{{\^e}}1  {î}{{\^i}}1 {ô}{{\^o}}1  {û}{{\^u}}1
      {Â}{{\^A}}1  {Ê}{{\^E}}1  {Î}{{\^I}}1 {Ô}{{\^O}}1  {Û}{{\^U}}1
      {œ}{{\oe}}1  {Œ}{{\OE}}1  {æ}{{\ae}}1 {Æ}{{\AE}}1  {ß}{{\ss}}1
      {ç}{{\c c}}1 {Ç}{{\c C}}1 {ø}{{\o}}1  {Ø}{{\O}}1   {å}{{\r a}}1
      {Å}{{\r A}}1 {ã}{{\~a}}1  {õ}{{\~o}}1 {Ã}{{\~A}}1  {Õ}{{\~O}}1
      {ñ}{{\~n}}1  {Ñ}{{\~N}}1  {¿}{{?`}}1  {¡}{{!`}}1
      {°}{{\textdegree}}1 {º}{{\textordmasculine}}1 {ª}{{\textordfeminine}}1
      % ¿ and ¡ are not correctly displayed if inconsolata font is used
      % together with the lstlisting environment. Consider typing code in
      % external files and using \lstinputlisting to display them instead.
}

%----------------------------------------------------------------------------------------
%	PRESENTATION INFORMATION
%----------------------------------------------------------------------------------------

\title[Linux kernel and C programming]{Linux kernel and C programming}
\subtitle{BLOCK 2: CMake and Pointers}
\author[Gergely Korcsák]{Gergely Korcsák}
\institute[OE NIK]{Óbuda University \\ \smallskip \textit{korcsak.gergely@nik.uni-obuda.hu}}
\date[\today]{\today}

%----------------------------------------------------------------------------------------

\begin{document}

%----------------------------------------------------------------------------------------
%	TITLE SLIDE
%----------------------------------------------------------------------------------------

\begin{frame}
	\titlepage
\end{frame}

%----------------------------------------------------------------------------------------
%	TABLE OF CONTENTS SLIDE
%----------------------------------------------------------------------------------------

\begin{frame}
	\frametitle{Presentation Overview} % Slide title, remove this command for no title

	\tableofcontents % Output the table of contents (all sections on one slide)
    %\tableofcontents[pausesections,sections={1}] % Output the table of contents (break sections up across separate slides)
\end{frame}

%----------------------------------------------------------------------------------------
%	PRESENTATION BODY SLIDES
%----------------------------------------------------------------------------------------

\begin{frame}
    \sft{CMake}
\end{frame}

\begin{frame}[fragile]
    \ssft{CMake variables}

    CMake variables also have type and scope. We can define them from the command line, and also in the CMake files.

    \begin{block}{Command line example}
        \begin{lstlisting}[style=CStyle]
$ cmake -S . -B __build \
    -DMY_STRING:STRING="hello"

\end{lstlisting}
    \end{block}

    \bigskip

    Or they can be defined in a CMake files.

    \begin{block}{CMakeLists.txt}
        \begin{lstlisting}[style=CStyle]
set(<variable> <value>
    [[CACHE <type> <docstring> [FORCE]] | PARENT_SCOPE])

\end{lstlisting}
    \end{block}

\end{frame}

\begin{frame}
    \ssft{CMake variable types and scopes}

    \begin{block}{Types}
        \begin{table}[h]
            \centering
            \begin{tabular}{|p{.32\textwidth}|p{.58\textwidth}|}
                \hline \textbf{Type}        & \textbf{Description}  \\
                \hline \ascode{FILEPATH}    & Exact file path \\
                \hline \ascode{PATH}        & Directory path \\
                \hline \ascode{STRING}      & Arbitrary string \\
                \hline \ascode{BOOL}        & Boolean ON/OFF \\
                \hline \ascode{INTERNAL}    & Persistent \\
                \hline
            \end{tabular}
        \end{table}
    \end{block}
\end{frame}

\begin{frame}[fragile]
    \frametitle{CMake variable types and scopes}

    \begin{block}{Scopes}
        By default every CMake variable is accessible from their local file, and in every other included file, from the
        same CMake file. We can chage the scopes by eahter setting it as \ascode{CACHE}-ed, or to \ascode{PARENT_SCOPE}

        \begin{table}[h]
            \centering
            \begin{tabular}{|p{.32\textwidth}|p{.58\textwidth}|}
                \hline \textbf{Scope}           & \textbf{Description}  \\
                \hline \ascode{CACHE}           & Make the variable accessible globally, after its file was included \\
                \hline \ascode{PARENT_SCOPE}    & Make the variable accessible to the file, which it is included from \\
                \hline
            \end{tabular}
        \end{table}
    \end{block}
        \begin{lstlisting}[style=CStyle]
#// <variable>  <value>       <type> <description>
set(MY_STRING   "hello" CACHE STRING "Hello string")

#// <variable>  <value> <scope>
set(MY_STRING   "hello" PARENT_SCOPE)

\end{lstlisting}

\end{frame}

\begin{frame}[fragile]
    \ssft{CMake conditionals}
    \begin{lstlisting}[style=CStyle]
if(<condition>)
  <commands>
elseif(<condition>) # optional block, can be repeated
  <commands>
else()              # optional block
  <commands>
endif()

\end{lstlisting}
\end{frame}
\begin{frame}[fragile]
    \frametitle{CMake conditionals}
    \begin{lstlisting}[style=CStyle]
if(NOT <condition>)
if(<cond1> AND <cond2>)
if(<cond1> OR <cond2>)

if((condition) AND (condition OR (condition)))

if(DEFINED <name>|CACHE{<name>}|ENV{<name>})
if(EXISTS <path-to-file-or-directory>)

if(<variable|string> MATCHES <regex>)
if(<variable|string> EQUAL <variable|string>)   # double
if(<variable|string> STREQUAL <variable|string>)

if ("/a//b/c" PATH_EQUAL "/a/b/c")

\end{lstlisting}
\end{frame}

\begin{frame}[fragile]
    \ssft{CMake include}

    We can eather add a subdirectory containing a \ascode{CMakeLists.txt} file, or include a .cmake file like the
    following:

    \begin{lstlisting}[style=CStyle]
.
|-- CMakeLists.txt <this file>
|-- my_config.cmake
`-- terminal
    |-- terminal.h
    |-- terminal.c
    `-- CMakeLists.txt

include(my_config.cmake)

add_subdirectory(terminal)

\end{lstlisting}


\end{frame}

%----------------------------------------------------------------------------------------

\begin{frame}
    \sft{Extendxed C}
\end{frame}

\begin{frame}[fragile]
    \ssft{Unions}

    Unions allow to store different data types in the same memory location.

    \begin{lstlisting}[style=CStyle]
struct http_request {
    uint32_t version;
    uint32_t protocol;
    uint32_t sender;
    uint32_t randomdata;
    uint8_t  data[0x1000];
};

union http {
    struct http_request header;
    uint8_t payload[0x2000];
};

\end{lstlisting}

\end{frame}
\begin{frame}[fragile]
    \frametitle{Unions}

    \begin{lstlisting}[style=CStyle]
int handle_payload(...) {
    union http req;

    req.header.version    = 1;
    req.header.protocol   = 0x34;
    req.header.sender     = IP_TO_INT32(192,168,1,12);
    req.header.randomdata = random();

    sprintf(req.header.data,
            "Hello from http payload...!!");

    eht_phy->send(req.payload,
                  4 * sizeof(uint32_t)
                  + strlen(req.header.data));
}

\end{lstlisting}

\end{frame}

\begin{frame}[fragile]
    \ssft{Function pointers}

    Where the fun begins\ldots

    \bigskip

    \begin{lstlisting}[style=CStyle]
// <return type> (*<name>)  (args...);
   int           (*sum)     (int, int);

\end{lstlisting}

\end{frame}
\begin{frame}[fragile]
    \frametitle{Function pointers}
    We can define function pointers as follows:
    \begin{lstlisting}[style=CStyle]
// First lets use the function
int calculate_sum(int a, int b) { return a + b; }

// And create a pointer
int (*sum)(int a, int b);

// Assigne it
sum = calculate_sum;

// Call it
int sum_of_ab = sum(12, 13);

\end{lstlisting}

\end{frame}

\begin{frame}[fragile]
    \ssft{Function pointers in structs}

    \begin{lstlisting}[style=CStyle]
struct command {
    char* name;
    void (*cmd)(int argc, char** argv);
};

\end{lstlisting}

\end{frame}

\begin{frame}[fragile]
    \ssft{Argc and argv}

    Command line arguments will follown the program separated by a space character.
    \begin{lstlisting}[style=CStyle]
$ terminal help

\end{lstlisting}
    In this example the terminal is the program, its arguments will be [terminal, help]. The first argument will always
    be the name of the application called.

    \bigskip

    \begin{lstlisting}[style=CStyle]
int main(int argc, char** argv) {
    int i;

    for (i = 0; i < argc; i++) {
        printf("arg[%i]: %s", i, argv[i]);
    }

    return 0;
}

\end{lstlisting}

\end{frame}

%----------------------------------------------------------------------------------------

\begin{frame}
    \sft{Excercises}
\end{frame}

\begin{frame}[fragile]
    \ssft{Read input arguments}

    Lets try out the following:

    \begin{lstlisting}[style=CStyle]
int main(int argc, char** argv) {
    int i;

    for (i = 0; i < argc; i++) {
        printf("arg[%i]: %s", i, argv[i]);
    }

    return 0;
}

\end{lstlisting}

\end{frame}

\begin{frame}[fragile]
    \ssft{Write a simple console application}

    \begin{lstlisting}[style=CStyle]
.
|-- CMakeLists.txt
|-- config.cmake
`-- userspace
    |-- functions
    |   |-- help.h
    |   |-- help.c
    |   |-- echo.h
    |   `-- echo.c
    |-- terminal.h
    |-- terminal.c
    `-- CMakeLists.txt

\end{lstlisting}
\end{frame}

%----------------------------------------------------------------------------------------
%	CLOSING SLIDE
%----------------------------------------------------------------------------------------

\begin{frame}[plain] % The optional argument 'plain' hides the headline and footline
	\begin{center}
		{\Huge The End}

		\bigskip\bigskip % Vertical whitespace

		{\LARGE Questions? Comments?}
	\end{center}
\end{frame}

%----------------------------------------------------------------------------------------

\end{document}
